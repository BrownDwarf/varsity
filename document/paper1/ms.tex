\documentclass[modern]{aastex631}
\bibliographystyle{aasjournal}

% \usepackage{fontspec}
% \usepackage[T1]{fontenc}
% \usepackage{newtxsf}
% \setmainfont{Fira Sans Book}[Scale=1.0]

\usepackage[caption=false]{subfig}


\begin{document}
\shorttitle{Condensate clouds with TESS and IGRINS}
\shortauthors{Gully-Santiago et al. }
\title{Disentangling brown dwarf clouds with IGRINS and contemporaneous TESS monitoring}

\author{Michael Gully-Santiago}
\affiliation{University of Texas at Austin Department of Astronomy}

\author{TBD}
\affiliation{TBD}


\begin{abstract}

We disentangle clouds on the nearby brown dwarf Luhman 16 using multi-epoch IGRINS with contemporaneous TESS monitoring.

\end{abstract}

\keywords{High resolution spectroscopy (2096)}

\section{Introduction}\label{sec:intro}

Here is an annotated bibliography.

\begin{deluxetable}{chc}
  \tablehead{
  \colhead{one} & \nocolhead{two} & \colhead{three}
  }
  \startdata
  \citet{2013ApJ...767L...1L} & - & Discovery paper\\
  \citet{2021ApJ...906...64A} & - & TESS data \\
  \citet{2016ApJ...825...90K} & - & HST Time-resolved spectroscopy maps\\
  \citet{2017MNRAS.470.1140B} & - & HST Orbit fitting\\
  \enddata
\end{deluxetable}


\section{Methodology}


\section{Data}

\subsection{IGRINS}
We acquired spectra from IGRINS \citep{park14,2018SPIE10702E..0QM} at Gemini South as part of Director's Discretionary Time proposal \texttt{GS-2021A-DD-104}.  We computed an angular separation $\rho = 1\farcs3 \pm 0 \farcs 1$ and $P.A. = 134 \pm 3^\circ$ based on the orbital parameters from Table 5 of \citet{2018A&A...618A.111L} for the epoch 2021.19; the uncertainties are based on the comparison of predictions from \citet{2017MNRAS.470.1140B} and \citet{2018A&A...618A.111L}, which came to different values at these small levels.  The $0\farcs34 \times 5\farcs0$ IGRINS slit at Gemini South presents an economic tradeoff on how to orient the slit relative to the binary.  Aligning the slit along the $P.A.$ collects more photons, but complicates the data reduction process.  Aligning the slit perpendicular to the $P.A.$ collects less flux per exposure, but allows the \texttt{plp} piepeline package \citep{jaejoonlee16} to be used for standard and well-tested data reduction.  We chose the latter strategy, aligning the IGRINS $P.A.$ to $44^\circ$, as shown in Figure \ref{fig:imaging}.

Luhman~16 appears near a local maximum of projected sky separation in 2021, with the B component expected to reside South and East of the A component.

\begin{figure*}[ht]
  \centering
    \includegraphics[width=3in]{figures/Luhman16_IGRINS_slit_20210311.pdf} \\
\caption{A $21'' \times 21''$ postage stamp---equal to the size of one TESS pixel---of the IGRINS $K-$band Slit Viewing Camera at Gemini South on March 11, 2021.  The color scale is inverted.  The PA of the AB components was intentionally aligned perpendicular to the $0\farcs34 \times 5\farcs0$ IGRINS slit seen as a white diagonal silhouette to the southeast (bottom left) of the binary. TESS Full Frame Images (not shown) and resulting lightcurves consist of the integrated light from both components.}
\label{fig:imaging}
\end{figure*}

The end!


\subsection{TESS}


\section{Results}
State the discoveries here.

\section{Discussion}

\subsection{Future observation strategies}
Luhman~16 will remain a benchmark target for the forseeable future, with hundreds more hours of valuable telescope resources plausibly destined for this single target.  Here we examine the tradeoffs in how future observation strategies could make the most judicious use of scarce resources moving forward.

Things to say:  
\begin{itemize}
  \item Slit alignment: How does alignment change with orbital separation
  \item The limits on the ability to calibrate the relative fluxes on the same slit
  \item Slit alignment: How the uncertainty in the orbital projection flows down to alignment
  \item The technical barrier of extracting two traces from the same echellogram, sky subtraction and PSF fitting (spectroastrometry)
  \item Point-and-stare versus snapshots and night-to-night calibration
  \item Spectral decomposition and 
\end{itemize}



\begin{acknowledgements}
Based on observations obtained at the international Gemini Observatory, a program of NSF’s NOIRLab, which is managed by the Association of Universities for Research in Astronomy (AURA) under a cooperative agreement with the National Science Foundation. on behalf of the Gemini Observatory partnership: the National Science Foundation (United States), National Research Council (Canada), Agencia Nacional de Investigaci\'{o}n y Desarrollo (Chile), Ministerio de Ciencia, Tecnolog\'{i}a e Innovaci\'{o}n (Argentina), Minist\'{e}rio da Ci\^{e}ncia, Tecnologia, Inova\c{c}\~{o}es e Comunica\c{c}\~{o}es (Brazil), and Korea Astronomy and Space Science Institute (Republic of Korea).

This work used the Immersion Grating Infrared Spectrometer (IGRINS) that was developed under a collaboration between the University of Texas at Austin and the Korea Astronomy and Space Science Institute (KASI) with the financial support of the Mt. Cuba Astronomical Foundation, of the US National Science Foundation under grants AST-1229522 and AST-1702267, of the McDonald Observatory of the University of Texas at Austin, of the Korean GMT Project of KASI, and Gemini Observatory.

This paper includes data collected by the TESS mission. Funding for the TESS mission is provided by the NASA's Science Mission Directorate.

The authors acknowledge the Texas Advanced Computing Center (TACC, \url{http://www.tacc.utexas.edu}) at The University of Texas at Austin for providing HPC resources that have contributed to the research results reported within this paper.
\end{acknowledgements}

\clearpage


\facilities{Gemini South (IGRINS), TESS}

\software{  pandas \citep{mckinney10, reback2020pandas},
  emcee \citep{foreman13},
  matplotlib \citep{hunter07},
  astroplan \citep{astroplan2018},
  astropy \citep{exoplanet:astropy13,exoplanet:astropy18},
  exoplanet \citep{exoplanet:exoplanet},
  numpy \citep{harris2020array},
  scipy \citep{jones01},
  ipython \citep{perez07},
  starfish \citep{czekala15},
  bokeh \citep{bokehcite},
  seaborn \citep{waskom14}}
  %pytorch \citep{NEURIPS2019_9015}} % No pytorch yet!


\bibliography{ms}


\clearpage

\appendix
\restartappendixnumbering

\section{Assumptions about cloud spectra} \label{appendix:clouds}

Here are some more details about how we describe clouds.
\end{document}
